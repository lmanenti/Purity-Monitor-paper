%**********************************************
\section{Electron drift velocity in LAr}
%**********************************************
To calculate the electron drift velocity in LAr, use the "ICARUS polynomial fit" for fields below 500 V/cm; use the "Walkoviac function for fields above 500 V/cm". I report only the ICARUS polynomial fit as we are never above 500 V/cm.

The ICARUS polynomial is as follows:
\begin{equation}
v(E) = (p_0 + p_1 E + p_2 E^2 + p_3 E^3 + 
    p_4 E^4 + p_5 E^5) \times K_2/K_1
\end{equation}  
where $E$ is in \textbf{kV/cm}, and $K_2$ and $K_1$ are defined as:
\begin{align}
& K_1 = p_0 + p_1 \widetilde{E} + p_2 \widetilde{E}^2 + p_3 \widetilde{E}^3 + 
   p_4 \widetilde{E}^4 + p_5 \widetilde{E}^5;\\
&K_2 = \left(w_1 (T - T_0) + 1\right) \left(w_3 \widetilde{E} \log[1 + w_4/\widetilde{E}] + w_5 \widetilde{E}^{w_6}\right) + 
   w_2 (T - T_0);
\end{align}
with $\widetilde{E} = 0.5$, $T=89$\,K, $T_0 = 90.371$\,K and the parameters $p_0 ... p_5, w_1...w_5 $ are:
\begin{equation}
\begin{aligned}
&p_0 = -0.03229;\\
&p_1 = 6.231;\\
&p_2 = -10.62;\\
&p_3 = 12.74;\\
&p_4 = -9.112;\\
&p_5 = 2.83;\\
\end{aligned}
\;\;\;\;\;\;\;\;
\begin{aligned}
&w_1 = -0.01481;\\
&w_2 = -0.0075;\\
&w_3 = 0.141;\\
&w_4 = 12.4;\\
&w_5 = 1.627;\\
&w_6 = 0.317;
\end{aligned}
\end{equation} 


The electron depletion then obeys the following differential equation:
\begin{equation}
dN/dt = -(\lambda_{el} N + \lambda_{life} N) = - (\lambda_{el} + \lambda_{life}) N
\end{equation}
with $\lambda_{el}=1/\tau_{el}$ and $\lambda_{life}=1/\tau_{life}$
the solution to this equation is the usual decreasing exponential:
\begin{equation}
N(t) = N_0 \exp{-\lambda t}
\end{equation}
where $\lambda = \lambda_{el} + \lambda_{life}$ corresponds to a ``compounded" decay rate. Since $\lambda$ is the inverse of the decay time, then we have that:
\begin{equation}
\lambda = \frac{1}{\tau} = \lambda_{el} + \lambda_{life} = \frac{1}{\tau_{el}} + \frac{1}{\tau_{life}}
\end{equation}
Thus, eqn \ref{eq:V_peak_final} becomes:
\begin{equation}
V_{peak} = V_{out} (t_d) = G_0 Q \frac{1-e^{-\frac{t}{(\frac{1}{\tau_{el}}+\frac{1}{\tau_{life}})^{-1}}}}{\frac{t_d}{(\frac{1}{\tau_{el}}+\frac{1}{\tau_{life}})^{-1}}}
\end{equation}

\begin{equation}
V_{out} (t) = \int dt' G(t-t') I(t') 
\end{equation}

\begin{equation}
G(t) = G_0 \Theta(t) e^{-\frac{t}{\tau{el}}}
\end{equation}

\begin{align}
 I(t) &=
  \begin{cases}
   \frac{Q v_e}{d}        & 0 \leq t \leq t_d \\
   0        			& t > t_d
  \end{cases}
 \end{align}

\begin{equation}
V_{out}(t) = G_0 Q \frac{
\left(e^{-\frac{t}{\tau_{el}}}-e^{-\frac{t}{\tau_{life}}}\right)
}{
\frac{t_d}{\left (\frac{1}{\tau_{life}}-\frac{1}{\tau_{el}}\right )^{-1}}
}
\end{equation}
